
\documentclass[portrait,a0paper,fontscale=0.25]{baposter}
\usepackage{geometry} % Added package for margin control
\geometry{top=1cm}    % Pushed content down by 2cm
\usepackage{wrapfig}

\usepackage[utf8]{inputenc}
\usepackage[T1]{fontenc}
\usepackage[sfdefault]{Fira Sans}
\usepackage[export]{adjustbox}

\usepackage{color}
\usepackage{graphicx}
\usepackage{amssymb,amsmath}
\usepackage[export]{adjustbox}

\renewcommand{\arraystretch}{1.5}

\usetikzlibrary{positioning}


\begin{document}

\color{black!80} % default font color
\begin{poster}{grid=false,
	eyecatcher=true,
	background=plain,
	bgColorOne=black!3, % background color
	columns=2,
	headerborder=none,
	textborder=none,
	headershape=rectangle,
	headershade=plain,
	boxshade=plain,
	boxColorOne=white,
	headershade=plain,
	headerColorOne=black!15, % box header background color
	headerFontColor=black
	}%
	{\includegraphics[height=7em]{logos/mff-black.pdf}}
	{Webová aplikace pro interaktivní
hlasování}
	{\vspace{1ex} Autor: Matěj Foukal | Vedoucí: Mgr. Petr Škoda, Ph.D.
}
	{\includegraphics[height=7em]{logos/uk-red.pdf}}



%
% LEFT COLUMN (CORRECTED)
%
\begin{posterbox}[column=0, name=motivace]{Motivace}
    Interaktivní hlasování umožňuje aktivizovat publikum a získat okamžitou zpětnou vazbu. 
    Současná komerční řešení jsou však často výrazně omezená ve svých bezplatných verzích. 
    Navíc je jejich kód uzavřený, což znemožňuje vlastní integrace s nástroji třetích stran. 

    Cílem této práce je vytvoření \textbf{open-source} webové aplikace určené pro hlasování, která vybraná omezení existujících řešení překonává.
\end{posterbox}

% Změna zde: Jednoduché umístění pod předchozí box
\begin{posterbox}[column=0, name=goals, below=motivace, headerColorOne=cyan!60, boxColorOne=cyan!20]{Cíle práce}
V rámci práce jsme identifikovali hlavní uživatelské role (učitel, student, správce) a na základě dotazníkového šetření jsme stanovili jejich potřeby. Z nich uvádíme několik příkladů:
\begin{itemize}
  \item Učitel: možnost vytvářet různé typy aktivit.
  \item Učitel: sdílet hlasování pomocí odkazu nebo QR kódu.
  \item Student: připojení k aktivitě bez nutnosti registrace.
  \item Správce: snadné nasazení systému.
\end{itemize}
\end{posterbox}

% Změna zde: Jednoduché umístění a oprava 'span'
% Parametr 'span' zde ani není nutný, box se natáhne sám.
\begin{posterbox}[column=0, name=architektura, below=goals]{Architektura systému}
    Při návrhu architektury jsme kladli důraz na oddělení webových aplikací pro studenty a učitele, 
    abychom umožnili jejich budoucí vizuální odloučení. 
    Server je realizován jako monolitická aplikace v~ASP.NET~a~C\#. 
    Data jsou ukládána v databázi PostgreSQL. Pro snadnou správu databáze jsme zvolili nástroj Adminer. Pro jednoduché spuštění a nasazení je systém připraven v~Dockeru.

    \begin{center}
        \includegraphics[width=0.64\textwidth]{img/structurizr_c4_l2 (1).png}
    \end{center}
\end{posterbox}


%
% RIGHT COLUMN
%
\begin{posterbox}[column=1, name=result1]{Ukázka uživatelského rozhraní}
    Uživatelské rozhraní bylo inspirováno webovou aplikací Slido. 
    Návrh probíhal ve více fázích, od prvotních návrhů rozhraní (wireframes) až po finální implementaci. 
    Důraz byl kladen především na jednoduchost, intuitivnost a responzivitu pro podporu různých zařízení. 
    \begin{center}
        \includegraphics[width=1\textwidth, valign=T]{img/user_docs_img_8.png}
    \end{center}
\end{posterbox}

% VÁŠ BOX S TESTOVÁNÍM (ZŮSTÁVÁ STEJNÝ)
\begin{posterbox}[column=1, name=sus_test, below=result1]{Ověření použitelnosti}
      Pro ověření použitelnosti systému byla zvolena standardizovaná metoda \textit{SUS} (System Usability Scale). 
    Testování se zúčastnilo 5 respondentů, kteří po splnění připravených úkolů ohodnotili 10 tvrzení o použitelnosti systému na škále 1 až 5.

    Výsledné skóre je vypočteno podle následujícího vzorce:
    $$ SUS = 2.5 \cdot (20 + \sum(\text{T}_{\text{liché}}) - \sum(\text{T}_{\text{sudé}})) $$

    Dosažený výsledek \textbf{81.5} bodů odpovídá \textbf{nadprůměrné použitelnosti}. Zároveň překračuje hranici 68 bodů, která značí průměrnou uživatelskou použitelnost.



    \large % Větší písmo pro výsledek
    \begin{center}
        Průměrné SUS skóre: \quad \bfseries\textcolor{green!60!black}{81.5 / 100}
    \end{center}
    \normalsize

\end{posterbox}

% OPRAVENO ZDE: Přidán parametr 'below=sus_test'
\begin{posterbox}[column=1, name=result3, below=sus_test, headerColorOne=green!60, boxColorOne=green!20]{Výsledek práce}
    \vspace{1ex}
Výsledkem této práce je funkční open-source systém, který splňuje klíčové stanovené uživatelské požadavky. 
    Představuje tak rozumnou alternativu k zavedeným řešením interaktivního hlasování a překonává některá z jejich omezení.
    \vspace{.5ex}
\end{posterbox}

\begin{posterbox}[column=1, name=conclusion, below=result3, 
  headerColorOne=orange!70, boxColorOne=orange!20]{Závěr a budoucí rozšíření}
    \begin{wrapfigure}{R}{0.25\linewidth} % R = doprava, šířka = 25% šířky boxu
        \centering
        \includegraphics[width=\linewidth]{img/github_qr.png}
    \end{wrapfigure}
Systém zatím nebyl nasazen ani testován v reálné výuce. 
V budoucnu proto plánujeme ověřit jeho výkonnost a nefunkční požadavky přímo mezi uživateli. 
Dále zvažujeme technologické úpravy, zejména přechod z \textit{pollingu} na \textit{WebSockets} pro skutečné real-time zobrazení výsledků. 
Za klíčové rozšíření považujeme také modul pro pokročilé generování statistik a reporting.

\end{posterbox}


\end{poster}
\end{document}
