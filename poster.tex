
\documentclass[portrait,a0paper,fontscale=0.25]{baposter}
\usepackage{wrapfig}

\usepackage[utf8]{inputenc}
\usepackage[T1]{fontenc}
\usepackage[sfdefault]{Fira Sans}

\usepackage{color}
\usepackage{graphicx}
\usepackage{amssymb,amsmath}
\usepackage[export]{adjustbox}

\renewcommand{\arraystretch}{1.5}

\usetikzlibrary{positioning}


\begin{document}

\color{black!80} % default font color
\begin{poster}{grid=false,
	eyecatcher=true,
	background=plain,
	bgColorOne=black!3, % background color
	columns=2,
	headerborder=none,
	textborder=none,
	headershape=rectangle,
	headershade=plain,
	boxshade=plain,
	boxColorOne=white,
	headershade=plain,
	headerColorOne=black!15, % box header background color
	headerFontColor=black,
	}%
	{\includegraphics[height=7em]{logos/mff-black.pdf}}
	{Webová aplikace pro interaktivní
hlasování}
	{\vspace{1ex} Matěj Foukal}
	{\includegraphics[height=7em]{logos/uk-red.pdf}}



%
% LEFT COLUMN (CORRECTED)
%
\begin{posterbox}[column=0, name=motivace]{Motivace}
    Interaktivní hlasování je cestou, jak aktivizovat publikum a získat okamžitou zpětnou vazbu.
    
    \vspace{1ex}
    Současná komerční řešení jsou však často:
    \begin{itemize}
        \item \textbf{Omezená} ve svých bezplatných verzích.
        \item \textbf{Placená}, což brání širšímu využití ve výuce.
        \item \textbf{Uzavřená} pro integraci s nástroji třetích stran.
    \end{itemize}
    \vspace{1ex}
    
    Cílem této práce je proto vytvoření \textbf{open-source alternativy}, která tato omezení překonává.
\end{posterbox}

% Změna zde: Jednoduché umístění pod předchozí box
\begin{posterbox}[column=0, name=goals, below=motivace, headerColorOne=cyan!60, boxColorOne=cyan!20]{Cíle práce}
    \begin{itemize}
        \item Navrhnout a implementovat otevřenou a flexibilní platformu.
        \item Umožnit snadné nasazení a další rozšiřování systému.
        \item Vytvořit přehledné UI pro učitele i studenty.
        \item Ověřit použitelnost výsledného řešení pomocí uživatelského testování.
    \end{itemize}
\end{posterbox}

% Změna zde: Jednoduché umístění a oprava 'span'
% Parametr 'span' zde ani není nutný, box se natáhne sám.
\begin{posterbox}[column=0, name=architektura, below=goals]{Architektura systému}
    Systém je navržen jako monolitická serverová aplikace (\textbf{ASP.NET Core}) poskytující API pro dvě oddělené klientské aplikace (\textbf{SvelteKit}). Ukládání dat zajišťuje databáze \textbf{PostgreSQL} a celý systém je kontejnerizován pomocí \textbf{Dockeru}.

    \vspace{2ex}
    \begin{center}
        \includegraphics[width=0.7\textwidth]{img/structurizr_c4_l2 (1).png}
    \end{center}
\end{posterbox}


%
% RIGHT COLUMN
%
\begin{posterbox}[column=1, name=result1]{Ukázka uživatelského rozhraní}
    % Zde doporučuji vložit 2-3 screenshoty vaší aplikace
    \begin{center}
        \includegraphics[width=1\textwidth, valign=T]{img/user_docs_img_8.png}
    \end{center}
\end{posterbox}

% VÁŠ BOX S TESTOVÁNÍM (ZŮSTÁVÁ STEJNÝ)
\begin{posterbox}[column=1, name=sus_test, below=result1]{Ověření použitelnosti (Metoda \textit{SUS})}
    Pro ověření použitelnosti systému byla zvolena standardizovaná metoda SUS. Testování se zúčastnilo 5 respondentů, kteří po splnění připravených úkolů ohodnotili 10 tvrzení na škále 1 až 5.

    Výsledné skóre bylo vypočteno podle následujícího vzorce:
    $$ SUS = 2.5 \cdot (20 + \sum(\text{T}_{\text{liché}}) - \sum(\text{T}_{\text{sudé}})) $$

        Dosažený výsledek značí \textbf{nadprůměrnou úroveň použitelnosti} a splňuje tak stanovený cíl projektu.


    \large % Větší písmo pro výsledek
    \begin{center}
        Průměrné SUS skóre: \quad \bfseries\textcolor{green!60!black}{81.5 / 100}
    \end{center}
    \normalsize

\end{posterbox}

% OPRAVENO ZDE: Přidán parametr 'below=sus_test'
\begin{posterbox}[column=1, name=result3, below=sus_test, headerColorOne=green!60, boxColorOne=green!20]{Výsledek}
    \large\bfseries
    \vspace{1ex}
    \begin{center}
        Výsledkem práce je funkční open-source systém splňující identifikované a stanovené uživatelské požadavky a očekávání.
    \end{center}
    \vspace{.5ex}
\end{posterbox}

\begin{posterbox}[column=1, name=conclusion, below=result3]{Závěry a budoucí práce}
    % --- Textová část nahoře ---
    \begin{wrapfigure}{R}{0.25\linewidth} % R = doprava, šířka = 25% šířky boxu
        \centering
        \includegraphics[width=\linewidth]{img/github_qr.png}
    \end{wrapfigure}
    \textbf{Klíčové vlastnosti:}
    \begin{itemize}
        \item Plně funkční systém dostupný jako \textbf{open-source}.
        \item Řešení postavené na moderních a rozšířených technologiích.
        \item Flexibilní architektura umožňující snadné přidávání nových typů aktivit.
    \end{itemize}
    \textbf{Budoucí práce:}
    \begin{itemize}
        \item Přechod z \textit{polling} na \textit{WebSockets} pro real-time aktualizace.
        \item Rozšíření o pokročilejší statistiky a generování reportů.
        \item Vytvoření dedikovaného rozhraní pro administrátory.
    \end{itemize}

\end{posterbox}


\end{poster}
\end{document}
